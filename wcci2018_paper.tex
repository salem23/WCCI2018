\documentclass[conference]{IEEEtran}
\IEEEoverridecommandlockouts
% The preceding line is only needed to identify funding in the first footnote. If that is unneeded, please comment it out.
\usepackage{cite}
\usepackage{amsmath,amssymb,amsfonts}
\usepackage{algorithmic}
\usepackage{graphicx}
\usepackage{textcomp}
\def\BibTeX{{\rm B\kern-.05em{\sc i\kern-.025em b}\kern-.08em
    T\kern-.1667em\lower.7ex\hbox{E}\kern-.125emX}}
\begin{document}

\title{Analyzing and Enhancing the Evaluation Process of Evolutionary Drivers for TORCS\\
{\footnotesize \textsuperscript{*}Note: Sub-titles are not captured in Xplore and
should not be used}
\thanks{This work has been supported in part by: Ministerio espa\~{n}ol de
Econom\'{\i}a y Competitividad under project TIN2014-56494-C4-3-P
(UGR-EPHEMECH), TIN2017-85727-C4-2-P (UGR-DeepBio) and TEC2015-68752 (also funded by FEDER).}
}

\author{\IEEEauthorblockN{1\textsuperscript{st} Mohammed Salem}
\IEEEauthorblockA{\textit{Department of Computer Sciences} \\
\textit{University of Mascara}\\
Mascara, Algeria \\
salem@univ-mascara.dz}
\and
\IEEEauthorblockN{2\textsuperscript{nd} Antonio Miguel Mora}
\IEEEauthorblockA{\textit{Department of Computer Sciences and Technology, ESIT} \\
\textit{International University of La Rioja (UNIR)}\\
Granada, Spain\\
antoniomiguel.mora@unir.net}
\and
\IEEEauthorblockN{3\textsuperscript{rd} Juan Julian Merelo}
\IEEEauthorblockA{\textit{Department of Architecture and Computer Technology} \\
\textit{University of Granada}\\
Granada, Spain \\
jmerelo@ugr.es}
\and
\IEEEauthorblockN{4\textsuperscript{th} Pablo Garc\'ia-S\'anchez}
\IEEEauthorblockA{\textit{Department of Computer Science, ESI} \\
\textit{University of C\'adiz}\\
C\'adiz, Spain \\
pablo.garciasanchez@uca.es}
}

\maketitle

\begin{abstract}
	
In Evolutionary Algorithms, designing a reliable and effective evaluation process for the individuals is a key aspect to care about. This process involves not only the design of a proper fitness function, but also the selection of the best solution for the optimization problem being solved; since this decision might not be easy when uncertainty or `noise' is present in the problem environment.
Automatically improving the behavioural engine of an autonomous driver for a car racing simulator such as TORCS, is an optimization problem which owns these difficulties. 
Thus, in this paper we describe an analysis and some proposals to improve the evaluation of optimized fuzzy drivers for TORCS. It starts from preliminary results obtained in previous papers and aims to obtain a more competitive autonomous driver.
To this end, two different fitness functions are studied in several experiments, along with two  
% Antonio - don't know how many approaches will we try
approaches for the selection of the best individual in the evolution, considering the noisy nature of the problem, which leads to the fact that the same individual could obtain very good or very bad results in the same races, and against the same opponents, due to the stochasticity present in the game and in the rivals' AI.
% Antonio - Complete this with the obtained resulst and conclusions
\end{abstract}
\begin{IEEEkeywords}
Videogames, Fuzzy Controller, TORCS, Steering control, Optimization, Genetic Algorithms
\end{IEEEkeywords}
%%%%%%%%%%%%%%%%%%%%%%%%%%%%%%%   INTRODUCTION   %%%%%%%%%%%%%%%%%%%%%%%%%%%%%%%
%
\section{Introduction}
\label{sec:intro}

Autonomous driving is a problem that shows up in many environments,
including the realm of self-driving cars but also in drones, ships,
trains and underwater vehicles. In general, there will be a series of
sensor inputs that include real speed, obstacles and other vehicles,
and based on those sensors, the bot will have to take a decision on
speed and steering that is optimal in several different
senses \cite{Autodriv2006}. Testing different autonomous driving methodologies in real
life is usually reserved to just a few big players, and methodologies
as well as algorithms are usually tested in simulated environments;
these simulated environments, at the same time, offer the incentive of
competition among your system and others. In this paper, we will be
using the Open Racing Car Simulator (TORCS) \cite{WebTORCS} is a very
realistic racing simulator which offers a great testbed for the
implementation and evaluation of autonomous drivers.  
It has been used several times for the celebration of Artificial
Intelligence (AI) competitions, where the aim is to create the best
autonomous driver for racings
\cite{torcs5,manualTORCS,Torcs3,torcs7,oponnents2010}. Besides being
able to test your car against other cars that have been published, it
can be used as a standalone environment to optimize driving in a
solo race. 


Evolutionary Algorithms (EAs) \cite{EAs_Back96} have been frequently applied in this scope, normally as an optimization tool to improve different kinds of behavioural engines \cite{Floreano2004,CarRacing_Pelta09,SAES2012,QuadfliegMO,torcs2012}, also including Fuzzy-based ones \cites{Guadarrama2008, LFAG, PerezEvolvingFuzzy09}.

Fuzzy Logic \cite{Fuzzy2011} represents a very good technique to define this kind of autonomous agents, since they model in part the human reasoning when driving.
For this reason the authors presented previously an approach combining two specialized fuzzy controllers able to decide the car's steering angle and desired speed at every single point (or tick) during a race \cite{evo17}. This driver were later improved \cite{evo18} optimizing the parameters of their membership functions by means of a Genetic Algorithm \cite{GAs_Goldberg89} which showed a more efficient behaviour.

% Antonio - Improve the motivation... I'm not inspired tonight. :_(
However, the proposed optimization method was a first step, which let the door open for many different ways of improvement. Thus, in this paper we propose and analyze different evaluation methods to create a more reliable and efficient evolutionary driver. First, two novel different fitness functions have been designed in order to find the best set of factors influencing in the performance of the driver during the race.
And Second,  other evaluation-related factors have been analyzed, such as the consideration of a track combining easy and difficult parts (i.e. curves).

Moreover, there is uncertainty/noise present in this environment, which means that the evaluation of the same individual/driver in the same conditions could yield very different results (due to the presence of stochastic factors). Thus, in order to better deal with this issue, we have also studied different approaches to choose the best individual of the every execution, i.e. the best driver.

% Antonio - comment on the results

%Once optimized, the best genetic-fuzzy based controllers (one per fitness) have been evaluated in a practice race (without rivals) first, and then in a real race against different drivers in TORCS. 
%According to the obtained results the enhanced controllers perform both much better than the original fuzzy controller, improving the lap time and reducing the received damage. Moreover they are much more competitive against tough rivals, reaching high ranks in the most difficult races.

% Antonio - describe the final paper structure
The rest of the paper is organized as follows. Next we present the
state of the art, to be followed by a description of the TORCS
simulator in Section \ref{sec:torcs} and the
method for optimizing fuzzy controllers will be presented in Section
Results will be presented next in Section \ref{sec:results}. Finally, conclusions and future lines of work will
be presented in section \ref{sec:conclusions}.


%%%%%%%%%%%%%%%%%%%%%%%%%%%%%%  STATE OF THE ART  %%%%%%%%%%%%%%%%%%%%%%%%%%%%%%
\section{State of the Art}
\label{sec:soa}

TORCS has become one of the main environments for research on AI since its launch in 2007 \cite{WebTORCS}. It offers different kind of problems to solve, such as the definition of the optimal parameters for the cars (before the race) and the main one which is the design of competitive autonomous drivers aiming to win races against other cars.

Evolutionary algorithms have targeted TORCS almost since its
publication, for instance, for determining the
optimal trajectory of a lap in a known circuit \cite{drivingGA2008},
but this approach suffers from the problem that the obtained
trajectory in the evolving process strongly depends on the initial
state of the car.  
In the same context, the authors in \cite{GaRaceLine2010} tried to design a novel approach to compute the optimal racing line without any human intervention, using a GA to find the best trade-off between
the minimization of two conflicting objectives: the length and
the curvature of the racing line.

However, definitely, the most prolific area of application of EAs
inside TORCS has been the optimization of autonomous controllers for
car driving, i.e. conducting a meta-optimization process. 
Thus, EAs have been applied to `refine' the parameters which define
the driver's behavior \cite{ButzCMAES09,SAES2012}, or to improve the
structure/architecture of the models \cite{evol,neurone}, working
offline, or online (during the game)
\cite{TanOnline08,Cardamone_Online_NN}.
% FUZZY + EAs
Our approach is focused in this line, proposing the application of an off-line genetic algorithm for the improvement of the parameters which determine the behavior of a controller for TORCS. We have focused on a Fuzzy-based model, as it is one of the best options for modeling human-like decisions and actions, as others authors have also used this kind of technique in the literature with good results \cite{torcs2012}. 
For instance, in \cite{Guadarrama2008}, a fuzzy rule-based car controller for a Car Racing Competition was built and tuned with co-evolutionary genetic algorithms. Two fuzzy sub-controllers were designed (acceleration and turning angle). %Pablo: in previous phrase you said one controller, not two
But this approach was applied to a simpler simulator than TORCS which is a more realistic and time-constrained simulator. 

Pérez et al. introduced an evolutionary fuzzy approach for TORCS in \cite{PerezEvolvingFuzzy09}, where they applied EAs for improving fuzzy models to infer the acceleration and turning angle. However, the models were not so specialized as the proposed here, since their controller did not compute the target speed, which is a key factor for a competitive controller. 

Onieva et al. \cite{LFAG} presented a parametrized modular architecture with a fuzzy system and a GA in the design of fuzzy logic controllers for steering wheel management that can reproduce human driver behavior, but it did not take the target speed into account, unlike our previous controller \cite{evo17} which computed the target speed and the steer with two fuzzy sub-controllers and whose membership functions parameters were defined by trial/error process.
In this paper, we propose to optimize these parameters using a real coded genetic algorithm aiming to improve the performance of the original fuzzy controller. %Pablo: I think this was the Evo* objective. Update this phrase to the new objective ("generalize results..." or whatever)

%%%%%%%%%%%%%%%%%%%%%%%%%%%%%%  TORCS  %%%%%%%%%%%%%%%%%%%%%%%%%%%%%%

\section{Experimental Setup}
\label{sec:torcs}

The Open Racing Car Simulator (TORCS) \cite{WebTORCS} is an open source, modern, multi-player, modular and portable racing simulator that allows users to race against computer-controlled opponents.
% Its high degree of modularity and portability, together with the realistic and real-time driving simulation, make it an ideal testbed for artificial intelligence research, as it can be seen in the literature (See Section \ref{sec:soa}). %Pablo: refer to previous section?
% The game offers different types of races from the practical single session to the championship.
%Mohammed ok could be removed 
There is a large set of sensors \cite{Torcs3} which the car can consider during a race, such as distances to track borders, to rivals, current fuel, current gear, position in the race, speed, or damage, among others.
% These sensors values are used by any TORCS driver bot to control the car by means of a set of actuators \cite{Torcs3}: the steering wheel `Steer', the accelerator `accel', the brake pedal and the gearbox.  %Pablo: use more large paragraphs instead short phrases.
% 	% Mohammed- ok
% We should rewrite this section. I don't know if there's anything new or anything learned from before, but we could include it here instead of just un-commenting - JJ
A controller is a program, running inside TORCS, that 
drives a car automatically. It gets as input information about the current state of
the car and its situation on the track (sensors). These collected data
are used to decide actions to perform in the next simulation tick.  


%*****************************  FUZZY CONTROLLER  ******************************
The controller proposed initially \cite{evo17} has the same modular
architecture as the simple TORCS driver; however, the target speed and
steering angle are computed by means of two modular and specialized
fuzzy sub-controllers, which consider five position sensors. This is
the controller which will be improved by means of a GA in this
work.

The {\em fuzzy target speed sub-controller} aims to estimate the
optimal target speed of the car, both in straight parts and curves of
the track, taking into account two criteria: move as fast as possible
and be safe. This estimation is based on two general cases: if the car
is in a straight line, the target speed will take a maximum value
(\textit{maxSpeed} km/h). However, if it is close to a curve, the
controller will decrease the current speed to a value included in the
interval \textit{[minSpeed, maxSpeed]} km/h. 

This fuzzy controller has an output, the speed, and three input values:
\begin{itemize}
	\item Front = Track\_9: front distance to the track border (angle 0º).  %Pablo: in PDF Track[9] seems a paper citation, change it to Track\_9 instead
	\item M5 = max (Track\_8, Track\_10): max distance to the track border in an angle of +5º and -5º with respect to Front.
	\item M10 = max (Track\_7, Track\_11): max distance to track border in an angle of +10º and -10º.
\end{itemize}

It is a Mamdani-based fuzzy system \cite{iancu2012} with three
trapezoidal Membership Functions (MF) for every input variable. The
description of these fuzzy inputs and output are represented in Table
\ref{tab:flouevar}. In \cite{evo17} we set the values by hand;
previously we had made initial tests using an evolutionary
algorithm. In this paper we will try to improve obtained results by
fine-tuning this evolutionary algorithm. %Pablo: extend this phrase, also explaining the validation experiments. For example "Moreover, we are also validating our approach by testing the evolved controllores in different types of tracks and against other controllers, to understand the obtained behaviour." or something like that

\begin{table*}
	\centering
	{\scriptsize
		\caption{Fuzzy variables description.}
		\label{tab:flouevar}
		\begin{tabular}{ |p{1.5cm}|p{2cm}|p{2cm}|p{2 cm}|p{1 cm}|p{1.5 cm}|p{1.5 cm}|}
			\hline
			{ \textbf{Variable}}&
			{ \textbf{Range}}&
			{ \textbf{Name}}&  
			{ \textbf{MF}} &
			{ \textbf{Low}} &
			{ \textbf{Medium}}&
			{ \textbf{High}} \\
			\hline
			Input & [0-100] m & Front & trapezoidal & [0-50] & [20-80] & [60-100]
			\\
			\hline
			Input & [0-100] m & M5 & trapezoidal &[0-40] & [10-70] & [50-100] 
			\\
			\hline
			Input & [0-100] m  & M10 & trapezoidal & [0-30] & [20-60] & [50-100]
			\\
			\hline 
			Output & [0-200] m/s & TargetSpeed & singleton & / & / & /
			\\
			\hline 
		\end{tabular} 
	}
\end{table*}

% This is from the last paper? If so, just a reference would be
% enough. We need to save space - JJ
% Consider whether adding this again or not -- JJ
% The base of rules was composed modeling the behavior of a human expert driver. Thus, this set is designed to maximize the car speed depending on the distance to the track border. The fuzzy rules are:


% \begin{itemize}
% {\small
% 	\item \texttt{IF Front is High THEN TargetSpeed is TS1}
% 	\item \texttt{IF Front is Medium THEN TargetSpeed is TS2}
% 	\item \texttt{IF Front is Low and M5 is High THEN TargetSpeed is TS3}
% 	\item \texttt{IF Front is Low and M5 is Medium THEN TargetSpeed is TS4}
% 	\item \texttt{IF Front is Low and M5 is Low and M10 is High THEN TargetSpeed is TS5}
% 	\item \texttt{IF Front is Low and M5 is Low and M10 is Medium THEN TargetSpeed is TS6}
% 	\item \texttt{IF Front is Low and M5 is Low and M10 is Low THEN TargetSpeed is TS7}\\
% }

% In addition, a crisp rule is added to obtain a maximum value of the target speed when the three input variables are as big as possible:\\
% {\small	
% \item \texttt{IF Front = MAXDISTSPEED or M5 = MAXDISTSPEED or M10 = MAXDISTSPEED THEN TargetSpeed = MAXSPEED}
% }
% \end{itemize}

% MAXDISTSPEED is the longest possible value for the track sensors, and MAXSPEED is the maximal speed for the specific car. 
% The output value is encoded by seven singletons TS1 to TS7, being respectively: 280, 240, 220, 180, 120, 60 and 30.\\


% %***********************************************
% \noindent
% \textbf{Fuzzy steering control sub-controller}\\
% %
% The second fuzzy controller aims to control the steering, estimating and determining the target position of the car. 

% The structure of this sub-controller is similar to the speed one, but, obviously, with the steering as output. Thus, the set of sensors considered is the same as in the speed case (in Table \ref{tab:flouevar}).

% Then, as general rules: if the car is in a straight line, it will set as target position half width of the race track (central position of the lane). Whereas, if the car is near a right curve, it will approach the path leading to the right, with a space between the car and the border of the track to avoid the loss of control. The same approach is considered if the car is near a left curve.

% In order to detect the curves, the controller focuses on the sensor values (M10, M5, and Front). So, if the value on Front sensor is the longest, there is a straight road; whereas if the values of M5 and M10 with positive angles (+5 and +10) are the longest, there is right curve; and the other way round.

% The base of rules has been defined again modeling the behavior of a human driver, so, for this controller:

% {\small
% \begin{itemize}		
% 	\item \texttt{IF Front is High THEN steer is S1}
% 	\item \texttt{IF Front is Medium AND M10 is High THEN  steer is S2}
% 	\item \texttt{IF Front is Medium AND M10 is Medium AND M5 is Medium THEN steer is S2}
% 	\item \texttt{IF Front is Medium AND M10 is Medium AND M5 is Low THEN steer is S3}
% 	\item \texttt{IF Front is Low AND M10 is High THEN steer is S3}
% 	\item \texttt{IF Front is Low AND M10 is Medium AND M5 is Medium THEN steer is S4}
% 	\item \texttt{IF Front is Low AND M10 is Medium AND M5 is Low THEN steer is S4}
% \end{itemize}	
% }

% The values for S1 to S4 are respectively: 0, 0.25, 0.5, and 1.
% When M10=Track[7] we will take negative values of the steer (steer=-steer).

% These controllers were defined with our own criteria, but they could be far from being optimal, so, in the following section we apply a Genetic Algorithm for their improvement.


%%%%%%%%%%%%%%%%%%%%%%%%%%%%  OPTIMISING WITH GAS  %%%%%%%%%%%%%%%%%%%%%%%%%%%%

% \section{Optimizing the fuzzy controllers with GA}
% \label{sec:GA_optimization}

% Designing an optimal fuzzy controller for TORCS racing needs a human expert to define the membership functions parameters and the rule base. This expert, even if he exists, could not provide an exact repartition of the fuzzy membership functions values over the universe of discourse. 

% This difficulty have led us to move towards the use of Genetic Algorithms  \cite{GAs_Goldberg89} because of their global exploration characteristic in a complex environment, as this problem plots. 

The proposed optimization approach aims to find the optimal parameters of the membership functions of the two sub-controllers previously introduced. 
The followed process is depicted in Figure \ref{fig:ga}, in which, as it can be seen, the GA uses TORCS for the evaluation of every individual during the evolutionary process.
\begin{figure}[!ht]
  \label{fig:ga}
  \begin{center}
    \includegraphics[width=10cm]{fig/flowchart}
  \end{center}
  \caption{Optimization of a fuzzy controller flowchart. The evaluation of an individual is performed by: putting the parameter values on the two sub-controllers, launching a race in TORCS with this configuration, obtaining the resulting values of Damage, Top Speed and mean Lap Time and using these values for the computation of the fitness of the individual.}
\end{figure}	
The GA starts by creating the initial population with random values
for the parameters in the defined range $[0,100]$. The fitness of each
candidate solution is computed by injecting its gene values to the
parameters of the membership functions of the two fuzzy
sub-controllers. The defined autonomous controller is used to drive a
car in a 20 laps race in a circuit without opponents, and the
results (Top speed, Damage and Mean Lap time) are used to compute the
fitness value. 

%***********************************************

As previously stated, the designed fuzzy controllers have trapezoidal membership functions given by Equation \ref{eq:trapmf}.
In such a controller, fuzzy rules are applied to linguistic
terms. These terms, which qualify a linguistic variable, are defined
through membership functions, which, in turn, depend on a set of
parameters that `describes' their shape (and operation). Using a GA we
will optimize the parameters of the membership functions that
constitute the fuzzy partition of the linguistic variable
\cite{ThangG08}. The input linguistic variables in our problem,
\textit{Front, Max5} and \textit{Max10}, are represented by three
trapezoidal membership functions (See Table \ref{tab:flouevar}). % Maybe reduce? I think these two sentences could be merged into one

% JJ - I have eliminated this too. It's a fuzzy conference, they
% should know what it's about.

A trapezoidal membership function in a finite universe of discourse \textit{[a, b]} can be defined by:

\begin{equation}
\mu_{A}(x)= \left \{
\begin{array}{ll}
\frac{x - x_{1}}{x_{2} - x_{1}},& x_{1} \leq x \leq x_{2}\\
1 , &x_{2} \leq x \leq x_{3}\\
\frac{x_{4} - x}{x_{4} - x_{3}},& x_{3} \leq x \leq x_{4}\\
0        ,& else\\	
\end{array}
\right.
\label{eq:trapmf}
\end{equation}
with:
\begin{equation}
x_{1} \leq x_{2} \leq x_{3} \leq x_{4}
\end{equation}
This MF function is defined by four parameters $x_{1}$, $x_{2}$,
$x_{3}$ and $x_{4}$ taking their values in the interval \textit{[a,
  b]}.% (See Figure \ref{fig:trapeze}).

\begin{figure}[!ht] 
	\begin{center}
		\includegraphics[scale=0.7]{fig/trapese}
		\caption {Trapezoidal MFs}
		\label{fig:trapeze}
	\end{center}
\end{figure}
And a fuzzy partition with \textit{n} trapezoidal membership functions
is defined by \textit{2n} variables (\textit{a =} $ x_{1}$,$x_{2}
$,. .., $x_{2n} $ \textit {= b})(Equation \ref{eq:e1}). In this case,
the representation is given by the
figure \ref{fig:at} 
\begin{figure}[!ht] 
	\begin{center}
		\includegraphics[scale=0.55]{fig/trapezoidal.png}
		\caption {Trapezoidal-shaped MFs coding}
		\label{fig:at}
	\end{center}
\end{figure}
with:
\begin{equation}
a = x_{1} \leq x_{2} \leq...\leq x_{2n-1} \leq x_{2n}=b 	
\end{equation}		

\begin{equation} 
\begin{tabular}{l}
$\mu_{A1}(x)=  \left \{
\begin{array}{ll}
1, &x_{1} \leq x \leq x_{2}\\
\frac{x_{3} - x}{x_{3} - x_{2}}, &x_{2} \leq x \leq x_{3}\\
0        , &x > x_{3}\\
\end{array} 
\right.$		\\ 	
$\mu_{Ai}(x)= \left \{
\begin{array}{ll} 
0, &x \leq x_{2i-2}\\
\frac{x - x_{2i-2}}{x_{2i-1} - x_{2i-2}}, &x_{2i-2} \leq x \leq x_{2i-1},n=2,...,i-1\\
1, & x_{2i-1} \leq x \leq x_{2i}\\
\frac{x_{2i+1} - x}{x_{2i+1} - x_{2i}},& x_{2i} \leq x \leq x_{2i+1}\\
0  , &x > x_{2i+1}\\
\end{array}  
\right.	$		\\
$\mu_{An}(x)= \left \{
\begin{array}{ll} 
0, &x \leq x_{2n-2}\\
\frac{x - x_{2n-2}}{x_{2n-1} - x_{2n-2}},& x_{2n-2} \leq x \leq x_{2n-1}\\
1 ,& x > x_{2n-1} 
\end{array} 
\right.$\\
\label{eq:e1}
\end{tabular}
\end{equation}

% As we have just seen, a linguistic variable is represented by a number
% of parameters that depend both on the number and type of used
% membership functions  \cite{ThangG08}. Also the choice of coding to
% use for these different parameters depends both on the desired
% precision on the values and on their range of values. 
% I don't think this paragraph is needed - JJ
%  Mohammed:  tell me that you left the results section , You haven't removed it  too JJ jeje

When the number of parameters is reduced and their ranges of variations are well defined, a GA with a binary coding is largely sufficient to find their optimal values. On the other hand, if the number of parameters becomes important, and their variation interval is not well known, the real coding is the most appropriate \cite{elsayed13}. 
Since our work requires some precision and the variation interval of
each parameter is not well known, we have considered a real coding
implementation in a vector that includes all variables to optimize.

In that GA, every individual is a vector of 18 values/parameters, 6 per variable. Figure \ref {fig:cromosome} illustrates the structure of the chromosome.
\begin{figure*}[!ht]	
  \begin{center}
    \includegraphics[width=12cm]{fig/chromosome2.png}
    \caption{Chromosome description}
    \label{fig:cromosome}	
  \end{center}	
\end{figure*}

The initialization of the chromosomes (first population) is performed
assigning random values inside a range of variation
\cite{GAs_Goldberg89}, in order to start from feasible values
\cite{evo17}. 
Tournament based selection has been used to elect chromosomes as
parents for genetic operators, while simple arithmetic two point
crossover \cite{crossGA2017} and non uniform mutation
\cite{mutation1997} have been chosen, as two of the most contrasted
methods in the literature. 

%***********************************************


The objective of the car controller is to win as many races as
possible. However, we have to optimize the most general case by
carrying out
solo {\em training races} in which we try to minimize the damage of
the car ($damage$) and the average lap time $LapTime$, while
maximizing $TopSpeed$. It is a multiobjective optimization problem,
but since we want to obtain a single controller, we will have to use
heuristics to derive two possible fitness functions:

\begin{description}
	\item[GFC1:]  
	\begin{equation} \label{fit1}
	\begin{array}{ll}
	f_{1} =   damage + &\alpha \cdot LapTime 
	\end{array}
	\end{equation}
	\item[GFC2:] 
	\begin{equation} \label{fit2}
	\begin{array}{lll}
	f_{2}= damage + &\alpha \cdot  LapTime + \beta \cdot \frac{1}{TopSpeed}
	\end{array}
	\end{equation}	
\end{description} %Pablo: I think the "Min" should not be before damage or LapTime, because we are minimizing the function f_1 in general. In fact, I would remove all the "Min" appearances.
%Mohammed : right

$\alpha$ and $\beta$ are two heuristic weights. The main difference is
in the use of the $TopSpeed$ in the fitness to enhance the controller performances in straight lines aiming to reduce lap time and the overall race time.
% There should be some explanation here why is that the particular
% parameter you have decided to change in the two tested fitnesses - JJ
% Moahmmed : Added
To evaluate the candidate controllers during the evolutionary process, we will make each of them compete in a 20 laps practice race in a medium difficulty circuit without rivals. We have omitted the presence of opponents in order to avoid including additional uncertainty sources to the optimization process.

Then, the obtained output values $damage$, $LapTime$ and $TopSpeed$ are collected to compute the corresponding fitness value. As a clarification, $LapTime$ is the average of the 20 laps time.


%%%%%%%%%%%%%%%%%%%%%%%%%%%%  RESULTS  %%%%%%%%%%%%%%%%%%%%%%%%%%%%

\section{Experimental results}  
\label{sec:results}

%This section is dedicated to the performance evaluation of our
%fuzzy-genetic controller, called \textit{FGC}. 

% This paragraph should maybe go to the previous section, because it's more "experimental setup" than results. - JJ	
We will first need to choose whose tracks and cars are going to be used in the
experiment among the ones TORCS provides; in our case, we have selected the E-Track5 circuit as it is a
quite complex one, with multiple turns. \textit{car1-tbr1} has been
selected as the driving car \cite{evo17}. According to previous
experiments, this is a fair choice due to its moderate
performance. This will lead our controller to be prepared to drive in the most usual conditions. 

We have evaluated the FGC with the two proposed fitness functions, %Pablo: I rewrote this paragraph completely
comparing them for racing performance. Also, we have  carried out two algorithm executions with two different population sizes: 20 and 50, respectively. The rest of the parameters are: Generations=50, Crossover rate=0.7, Mutation rate=0.3, number of runs per configuration = 20.

%shown in Table \ref{tab:GA_config}: 
%	
\begin{itemize}
	\item \textbf{GFC1}: GA-Fuzzy controller with fitness 1 (Equation \ref{fit1}).
	\item \textbf{GFC2}: GA-Fuzzy controller with fitness 2 (Equation \ref{fit2}).
\end{itemize}

\begin{table}[!ht]	
		\centering
{\scriptsize
		\caption{GA parameters}
		\label{tab:GA_config}
		\begin{tabular}{|p{3.6cm}|p{3cm}|}
			\hline \textbf{Population size} & 20 \\
			\hline \textbf{Generations} & 50   \\
			\hline \textbf{Crossover rate$\textit{P}_{\textit{c}}$} &  0.7 \\
			\hline \textbf{Mutation rate $\textit{P}_{\textit{m}}$} &  0.3   \\ 		
			\hline          
		\end{tabular}	
}
\end{table}

The coefficients $\alpha$ and $\beta$ are chosen to be $1$  and
$10*MaxSpeed$ respectively, where $MaxSpeed $ is the maximum value of
speed that \textit{car1-tbr1}  could take ($MaxSpeed=300$)
\cite{evo17}, this choice is motivated by the fact to normalize
the Top speed values and make them in the same level as other fitness
terms. 
The results of these runs are shown in Table \ref{tab:runsresults}.
\begin{table*}[ht]
	\centering
	{\scriptsize
		\caption{ Results of 10 runs of GA with the two fitness
			functions. Please bear in mind that fitness follow different
			formula, and thus cannot be compared; LapTime and Damage
			should be the quantities used for comparison. }
		{
			\begin{tabular}{|c||c|c|c|c||c|c|c|c|}
				\hline
				&\multicolumn{8}{c||}{ with population size 20}\\ %Pablo: Remove EVO and put a reference to it, instead
				%Mohammed could we cite evo before  the final version?, I just put EVO  to explain that theese data is our previous work
				
				\hline
				&\multicolumn{4}{c||}{GFC1}&\multicolumn{4}{c|}{GFC2}\\
				\cline{2-9}
				& Min GFC1 & $LapTime$ & $Damage$ & $TopSpeed$& Min fit. 2& $LapTime$ & $Damage$ & $TopSpeed$\\
				\hline
				% 1 &46.42& 33.42& 13& 43.8005 & 30.98& 0& 234\\
				
				% 2 &31.09& 31.09& 0& 51.6614& 29.87& 11& 278\\
				
				Best &\textbf{29.44}& 29.44& 0&231&\textbf{39.74} & 29.25& 0& 286\\
				
				% 4 &31.12& 31.12& 0& 60.5155& 31.64& 16& 233\\
				
				% 5 &39.63& 31.63&8&  41.7232 & 30.05& 0& 257\\
				
				% 6 &30.21& 30.21& 0&  41.8588 & 30.14& 0& 256\\
				
				% 7 &29.73& 29.73& 0& 40.3140 &29.12& 0& 268\\
				
				% 8 &37.29& 31.29&6& 40.8510 & 30.75& 0& 297\\
				
				% 9 &34.07& 30.07& 4 & 40.8596 &29.99& 0& 276\\
				
				% 10&29.88& 29.88& 0& 40.0467&29.63& 0& 288\\
				\hline
				Mean &33.88 &30.79 &3.10&227.43&   44.14&  30.14 &   2.70&267.30\\ 
				St. Dev.&5.61&1.18&4.58&32.55&6.73&0.78 &5.81  & 22.03\\
				\hline
				\hline
				&\multicolumn{8}{c||}{ with population size 50}\\
				\hline
				&\multicolumn{4}{c||}{GFC1}&\multicolumn{4}{c|}{GFC2}\\
				\cline{2-9}
				& Min GFC2 & $LapTime$ & $Damage$ & $TopSpeed$& Min fit. 2& $LapTime$ & $Damage$ & $TopSpeed$\\
				\hline
				Best &\textbf{28.78}& 28.78& 0&233&\textbf{28.11} & 38.52& 0& 288\\
				\hline
				Mean &33.14 &29.89 &4.14&230.19&   42.93&  29.57 &   3.46&271.74\\ 
				St. Dev.&4.98&1.32&4.22&31.14&5.63&0.93 &5.19  & 23.90\\
				\hline
				
			\end{tabular}
		}\label{tab:runsresults}
	}
\end{table*}
Wilcoxon rank sum non-parametric test is used to reject or accept the
null hypothesis of equality of medians of the values of the two
fitness functions for the 20 runs with 50 chromosomes. The obtained
p-value was $p =  0.0011$, this result lead to the rejection of null
hypothesis with a threshold $\alpha=0.01$ which allows us to conclude
that the two samples sets are different. 

Since the \textbf{GFC2} controller also optimizes the TopSpeed, we can
notice that it is clearly superior to that of \textbf{GFC1} which
surely influences the overall Lap time. This improvement in TopSpeed
greatly increases the performance of the the \textbf{GFC2}.

Increasing the population size to 50  has led to better values of the
two controllers, yielding better values for lap time, speed and
damage. This increase in the population size has allowed a better
coverage of the research space, thus getting closer to the optimal
solution.  
The best solution obtained with each fitness function from these runs
will be used in races against selected opponents. 
% We will have to include the new membership functions here - JJ
% We represented the resulted membership functions of these two optimal individuals considering the different fitness functions in Figures \ref{fig:mffront}, \ref{fig:mfmax5} and \ref{fig:MFMAX10}.
% \begin{figure}%
% 	\centering
% 	\subfigure[with Fitness 1]{%
% 		\label{fig:front1}%
% 		\includegraphics[width=0.6\textwidth,height=3cm]{fig/MFFRONT}}%
% 	\subfigure[with Fitness 2]{%
% 		\label{fig:front2}%
% 		\includegraphics[width=0.6\textwidth,height=3cm]{fig/MFFRONT2}}%
% 	\caption{Front input MFs with GA}
% 	\label{fig:mffront}
% \end{figure}
% %
% \begin{figure}%
% 	\centering
% 	\subfigure[with Fitness 1]{%
% 		\label{fig:fmax51}%
% 		\includegraphics[width=0.6\textwidth,height=3cm]{fig/MFFMAX5}}%
% 	\subfigure[with Fitness 2]{%
% 		\label{fig:fmax52}%
% 		\includegraphics[width=0.6\textwidth,height=3cm]{fig/MFFMAX52}}%
% 	\caption{Max5 input MFs with GA}
% 	\label{fig:mfmax5}%
% \end{figure}
% %
% \begin{figure}%
% 	\centering
% 	\subfigure[with Fitness 1]{%
% 		\label{fig:MFMAX101}%
% 		\includegraphics[width=0.6\textwidth,height=3cm]{fig/MFMAX10}}%
% 	\subfigure[with Fitness 2]{%
% 		\label{fig:MFMAX102}%
% 		\includegraphics[width=0.6\textwidth,height=3cm]{fig/MFMAX102}}%
% 	\caption{Max10 input MFs with GA}
% 		\label{fig:MFMAX10}%
% \end{figure}
%
The shapes of the obtained membership functions are completely different from
those obtained by Trial/Error in the previous work \cite{evo17}
where the Medium linguistic variable of the new functions has bigger
range. This makes the controller very sensitive to the middle
distances of the inputs, like for a real driver who considers most of
the cases the car distance from the borders in that range. 
The other remark from the obtained  membership functions is the
dimension of the common range between the LOW and MEDIUM, which
provide a higher diversity in the output values. 


%%%%%%%%%%%%%%%%%%%%%%%%%%%%%%%%%%%%%%%%%%%%%%%%%%%%%
%Pablo: I suggest create subsection for each new experiment for validation, and explain before its purpose.
The two best genetic based fuzzy controllers obtained in the previous
experiments, one per fitness function, and thus named $GFC1$ and
$GFC2$, have been tested
in a practice race together. They were run %Pablo: do not use future,
                                %if you are using past tense stick
                                %with it (past perfect can also be
                                %used)
% Done - JJ
each one for 20 laps in E-Track5 circuit, which was the one used during the
evolution; then, they will be tested also in a practice race in
E-Road, a track not used previously. The obtained results are presented
in Table \ref{resultat20}. 

\begin{table}[!ht]
	\centering
	{\scriptsize
		\caption{Results of the three controllers in a 20 laps
			practice race. Results of the $AD$ controller
			\cite{evo17}, a hand-designed fuzzy controller, are
			included for comparison purposes where tested.}
		\label{resultat20}
		\begin{tabular}{|p{3cm}|c|c|c|}
			\hline
			\multicolumn{4}{|c|}{\textbf{E-Track 5}}  \\
			\hline \textbf{Results} & \textbf{AD} & \textbf{GFC1} & \textbf{GFC2}\\
			\hline Best Lap Time         & 29:70 & 30:01 & 29:21 \\
			\hline Top Speed          & 209 & 224 & 234\\
			\hline Min Speed          & 168 & 151 & 186 \\
			\hline Damage          & 936 & 0 & 0\\
			\hline
		\end{tabular}
		\begin{tabular}{|p{3cm}|c|c|}
			\hline
			\multicolumn{3}{|c|}{\textbf{CG Track2}}  \\ 
			\hline \textbf{Results} & \textbf{GFC1} & \textbf{GFC2}  \\
			\hline Best Lap Time         &       01:04:96     & 01:02:32  \\
			\hline Top Speed &220  &  238 \\
			\hline Min Speed               & 33             & 40  \\
			\hline Damage                    & 0              & 0  \\
			\hline 
		\end{tabular}
	}
\end{table} %PABLO: why AD is not compared in CG Track2?

From the table, we can see that the fuzzy controllers optimized by the
GA yield the best results, obtaining very good overall global race
times and eliminating damage, which is reduced to 0. For the sake of
comparison, we include the hand-designed AD controller \cite{evo17},
which finished the practice race where it was tested with a lot of
damage, implying that it could, in some difficult cases, not finish
the race. 
Testing the controllers in CG Track2, which is quite long and
difficult as it can be seen by the time it takes to run a single lap,
has proved their value in the adaptation to other tracks different
from the one used for `training', that is, the optimization of the
fuzzy controllers. 

The GFC2 controller has run with a higher speed (considering overall Top
Speed and Min Speed) than GFC1 in the two tracks. This is a positive
consequence of the inclusion of the $TopSpeed$ variable in the fitness
computation so the GA based fuzzy controller has optimized the speed
of the car due to early braking and detection of turns and their
curving angles. This ability  of the GA-fuzzy controller collaborates
to minimize the overall race time and thus the final ranking.
According to these results, GFC2 seems to be the best controller.

%%%%%%%%%%%%%%%%%%%%%%%%%%%%%%%%%%%%%%%%%%%%%%%%%%%%%
% You can't simply drop the reader into the next experiment. It has to
% be "narrated", linked to the previous paragraphs and justified. For
% instance - jj
% This still applies - JJ
Comparing average lap time gives us an overall idea of which
controller performs the best; however, at the end of the day in a
racing game the race has to be won. That is why we have tested every
fuzzy separately from the others in a real race against five standard controllers from each team integrated with TORCS. Tables \ref{tab:gfc1real} and \ref{tab:gfc2real}   illustrate their performance in two 5 laps real races. 
\begin{table*}[!ht]
	\centering
	{\scriptsize
		\caption{Results of GFC1 in two real races (5 laps)}
		% Explain the "Race time" row - JJ
		\label{tab:gfc1real}
		\begin{tabular}{|p{2 cm}|p{1.5 cm}|p{1.5 cm}|p{1.5 cm}|p{1.5 cm}|p{1.5 cm}|p{1.5 cm}|}
			\hline \textbf{E-TRACK5} &   \textbf{GFC1} & \textbf{berwin 10} & \textbf{bt 3} &\textbf{damned 2} & \textbf{inferno 5} & \textbf{tita 10}  \\
			\hline \textbf{Ranking} & 3/6&4/6&1/6&5/6&2/6&6/6\\			
			\hline \textbf{Race Time}	& 02:29:32\newline+24:11&02:29:32\newline +1 lap &02:29:32&02:29:32\newline +1 lap &02:29:32\newline+13:67&02:29:32\newline+1 lap\\	
			\hline \textbf{Best Lap}&33:79& 35:39&28:09&36:73&31:49&34:12\\	
			\hline \textbf{Max Speed}& 199&206&233&198&229&219\\	
			\hline \textbf{Damages}& 0&0&0&599&7&566 \\	
			\hline 
		\end{tabular}
		
		\begin{tabular}{|p{2 cm}|p{1.5 cm}|p{1.5 cm}|p{1.5 cm}|p{1.5 cm}|p{1.5 cm}|p{1.5 cm}|}
			\hline \textbf{CG Track2} & \textbf{GFC1}&\textbf{berwin 10} & \textbf{bt 3} &\textbf{damned 2} & \textbf{inferno 5} & \textbf{tita 10}  \\
			\hline \textbf{Ranking} & 3/6&4/6&1/6&6/6&5/6&2/6\\			
			\hline \textbf{Race Time}	&  05:10:66\newline +25:43&  05:10:66\newline+55:65& 05:10:66& 05:10:66\newline+1 lap& 05:10:66\newline+38:44& 05:10:66\newline+19:82\\	
			\hline \textbf{Best Time}& 1:03:65 &1:04:21&1:00:57&1:04:26&1:03:19&1:03:98\\	
			\hline \textbf{Max Speed}& 233&236&288&200&238&229\\
			\hline \textbf{Damage}& 112& 376 & 433&988&541&890\\	
			\hline 
		\end{tabular}
	}
\end{table*}


\begin{table*}[!ht]
	\centering
	{\scriptsize
		\caption{Results of GFC2 in two real races (5 laps)}
		\label{tab:gfc2real}
		\begin{tabular}{|p{2 cm}|p{1.5 cm}|p{1.5 cm}|p{1.5 cm}|p{1.5 cm}|p{1.5 cm}|p{1.5 cm}|}
			\hline \textbf{E-TRACK5} & \textbf{GFC2}&\textbf{berwin 10} & \textbf{bt 3} &\textbf{damned 2} & \textbf{inferno 5} & \textbf{tita 10}  \\
			\hline \textbf{Ranking} & 2/6&4/6&1/6&6/6&3/6&5/6\\			
			\hline \textbf{Race Time}	& 02:30:83\newline +03:99&  02:30:83\newline+1 lap&02:30:83&02:30:83\newline+1 lap&02:30:83\newline+08:35&02:30:83\newline+1 lap\\	
			\hline \textbf{Best Time}& 29:82 &36:38&28:35&37:04&30:53&36:00\\	
			\hline \textbf{Max Speed}& 214&202&230&188&226&204\\	
			\hline \textbf{Damage}& 0& 0 & 343&1230&0&668\\	
			\hline 
		\end{tabular}
		
		\begin{tabular}{|p{2 cm}|p{1.5 cm}|p{1.5 cm}|p{1.5 cm}|p{1.5 cm}|p{1.5 cm}|p{1.5 cm}|}
			\hline \textbf{E-ROAD} & \textbf{GFC2}&\textbf{berwin 10} & \textbf{bt 3} &\textbf{damned 2} & \textbf{inferno 5} & \textbf{tita 10}  \\
			\hline \textbf{Ranking} & 3/6&4/6&1/6&6/6&2/6&5/6\\			
			\hline \textbf{Race Time}	& 05:38:23\newline +17:72&  05:38:23\newline+1 lap&05:38:23&05:38:23\newline+1 lap&05:38:23\newline+10:73&05:38:23\newline+1 lap\\	
			\hline \textbf{Best Time}& 1:17:34 &1:16:29&1:14:97&1:20:80&1:13:98&1:15:29\\	
			\hline \textbf{Max Speed}& 221&206&228&178&228&206\\
			\hline \textbf{Damage}& 120& 356 & 753&2750&130&894\\	
			\hline 
		\end{tabular}
	}
\end{table*}

$GFC1$ and $GFC2$ controllers are quite competitive in these races; 
GFC2 has got an excellent second position in the track used during optimization (E-Track), and it has also got a remarkable third rank in the unknown track (E-Road). 
Both controllers have dealt very well for not being damaged, which even the winner, {\tt bt 3}, could not avoid. 

These results are a confirmation of the proper optimization done by the GA and mainly when the Top Speed was considered in the fitness. 
The obtained results in real races with opponents from tough teams of TORCS are encouraging even if the optimization process was in practice races.
This good adaptation of the proposed controller in races with rivals is due to the fact the modular fuzzy controller takes into consideration the presence of opponents in the track \cite{evo17}. 
The enhancement of that driver by the optimal values of the membership function values, allows it to detect and overtake the other cars with no damage or stuck.

In the last experiment, we tried to get the best of our controller by testing its limits in a disadvantageous track, so A-Speedway was selected. 
The results are shown in Table \ref{tab:gfc2real2} where one
could clearly notice the degradation of the performances of the
optimized fuzzy controller. It was ranked in the fourth position just before {\tt tita10} and {\tt bt3} controllers. Its top speed is acceptable considering that the used car is not as fast as others but the fast lap time was higher and the GFC2 controller was not as competitive with its rivals in this race. 

These results could be justified by the nature of the track. A-Speedway is an oval circuit with only four turns, this kind of tracks deprives our controller of its strongest point which is the late braking and the selection of an optimal trajectory in turns. 
The fact that the car has a worse average speed compared to the others and that our controller can not compensate for this loss of time in the turns will not work in favor of getting better best lap times.  

\begin{table*}[!ht]
	\centering
	{\scriptsize
		\caption{Results of GFC2 in a real race (5 laps)}
		\label{tab:gfc2real2}
		\begin{tabular}{|p{2 cm}|p{1.5 cm}|p{1.5 cm}|p{1.5 cm}|p{1.5 cm}|p{1.5 cm}|p{1.5 cm}|}
			\hline \textbf{A-Speedway} & \textbf{GFC2}&\textbf{berwin 10} & \textbf{bt 3} &\textbf{damned 2} & \textbf{inferno 5} & \textbf{tita 10}  \\
			\hline \textbf{Ranking} & 4/6&2/6&1/6&6/6&3/6&5/6\\			
			\hline \textbf{Race Time}	& 02:37:74\newline +29:12&  02:37:74\newline +19:03&02:37:74&02:37:74\newline+1 lap&02:37:74\newline+28:89&02:37:74\newline+1 lap\\	
			\hline \textbf{Best Time}& 35:78 &32:83&29:40&39:49&34:86&40:92\\	
			\hline \textbf{Max Speed}& 239&251&266&223&259&238\\	
			\hline \textbf{Damage}& 119& 615&1290 &363&739&899\\	
			\hline 
		\end{tabular}
	}
\end{table*} 


%%%%%%%%%%%%%%%%%%%%%%%%%%%%  CONCLUSIONS  %%%%%%%%%%%%%%%%%%%%%%%%%%%%
\section{Conclusions and Future Work} 
\label{sec:conclusions}

In this work, we have presented an improved Genetic Algorithm implementation that optimizes and improves an autonomous driver using fuzzy systems for TORCS simulator \cite{evo17}. It combines two sub-controllers, one to calculate the target speed and the other for the direction, that is, for driving the steering wheel.

After initial tests, that showed the promise of using evolutionary algorithms with two different fitness functions, one considering the average lap time and the car damage and another adding the top speed reached, we have fine-tuned some algorithm parameters to obtain better results. %PABLO: "Also, in order to validate the obtained controllers, different experiments have also been carried out on tracks not used for training ."

The yielded results are very promising since the optimized controllers (one per fitness function) were ranked among the first ones in three different evaluation races with rivals, with the minimum of damage.

In the comparison with the original (before the optimization) fuzzy controller, the improvement can be clearly seen in the results. The new controllers are able to drive much faster than it, and moreover they manage to not receiving any damage, while the original controller even crashed the car in some races.

The results show that including the top speed in the calculus improves
results, since the obtained drivers have proved to be able to run a 10
to 15\% faster in the races. However the damage term must be also
considered to `compensate' somehow the influence of the top speed,
otherwise the controller would be extremely aggressive and would not
finish many of the races.

Thus, we can conclude from the results that the presented evolutionary
algorithm with the
proposed fitness functions are well suited for finding the best
trade-off between the two objectives of any racing controller: damage
and speed. 

% We have to revise this. 
% Antonio - revised and rewritten
Nevertheless, these results can be improved by extending the
evaluation of population controllers in the Genetic algorithm to other
tracks and not just one, to allow the elected controller to adapt to many different situations during the races.
The applied GA could be improved in different ways, for instance, reducing its computation time by means of the parallelization of the evaluation phase.
Also, a multi-objective approach could be implemented, in which the main objectives to address by the controller could be optimized at once.
Moreover, we could also try to generate, optimize and tune
automatically the rule base of the fuzzy controller by means of a
Genetic Programming algorithm.  

Finally, the fuzzy controller could be evolved and adapted to be an efficient autonomous driver for a real car. This could be addressed by considering real-life traffic situations instead of races and, of course, redefining the fitness functions to accomplish other objectives, mainly related with security and comfort.

\section*{Acknowledgments}

This work has been supported in part by: Ministerio espa\~{n}ol de
Econom\'{\i}a y Competitividad under project TIN2014-56494-C4-3-P
(UGR-EPHEMECH), TIN2017-85727-C4-2-P (UGR-DeepBio) and TEC2015-68752 (also funded by FEDER).


\bibliographystyle{IEEEtranS}
\bibliography{fuzzy_torcs}

\end{document}
